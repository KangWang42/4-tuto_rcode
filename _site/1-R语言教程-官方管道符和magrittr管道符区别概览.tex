% Options for packages loaded elsewhere
\PassOptionsToPackage{unicode}{hyperref}
\PassOptionsToPackage{hyphens}{url}
%
\documentclass[
]{article}
\usepackage{amsmath,amssymb}
\usepackage{iftex}
\ifPDFTeX
  \usepackage[T1]{fontenc}
  \usepackage[utf8]{inputenc}
  \usepackage{textcomp} % provide euro and other symbols
\else % if luatex or xetex
  \usepackage{unicode-math} % this also loads fontspec
  \defaultfontfeatures{Scale=MatchLowercase}
  \defaultfontfeatures[\rmfamily]{Ligatures=TeX,Scale=1}
\fi
\usepackage{lmodern}
\ifPDFTeX\else
  % xetex/luatex font selection
\fi
% Use upquote if available, for straight quotes in verbatim environments
\IfFileExists{upquote.sty}{\usepackage{upquote}}{}
\IfFileExists{microtype.sty}{% use microtype if available
  \usepackage[]{microtype}
  \UseMicrotypeSet[protrusion]{basicmath} % disable protrusion for tt fonts
}{}
\makeatletter
\@ifundefined{KOMAClassName}{% if non-KOMA class
  \IfFileExists{parskip.sty}{%
    \usepackage{parskip}
  }{% else
    \setlength{\parindent}{0pt}
    \setlength{\parskip}{6pt plus 2pt minus 1pt}}
}{% if KOMA class
  \KOMAoptions{parskip=half}}
\makeatother
\usepackage{xcolor}
\usepackage[margin=1in]{geometry}
\usepackage{color}
\usepackage{fancyvrb}
\newcommand{\VerbBar}{|}
\newcommand{\VERB}{\Verb[commandchars=\\\{\}]}
\DefineVerbatimEnvironment{Highlighting}{Verbatim}{commandchars=\\\{\}}
% Add ',fontsize=\small' for more characters per line
\usepackage{framed}
\definecolor{shadecolor}{RGB}{248,248,248}
\newenvironment{Shaded}{\begin{snugshade}}{\end{snugshade}}
\newcommand{\AlertTok}[1]{\textcolor[rgb]{0.94,0.16,0.16}{#1}}
\newcommand{\AnnotationTok}[1]{\textcolor[rgb]{0.56,0.35,0.01}{\textbf{\textit{#1}}}}
\newcommand{\AttributeTok}[1]{\textcolor[rgb]{0.13,0.29,0.53}{#1}}
\newcommand{\BaseNTok}[1]{\textcolor[rgb]{0.00,0.00,0.81}{#1}}
\newcommand{\BuiltInTok}[1]{#1}
\newcommand{\CharTok}[1]{\textcolor[rgb]{0.31,0.60,0.02}{#1}}
\newcommand{\CommentTok}[1]{\textcolor[rgb]{0.56,0.35,0.01}{\textit{#1}}}
\newcommand{\CommentVarTok}[1]{\textcolor[rgb]{0.56,0.35,0.01}{\textbf{\textit{#1}}}}
\newcommand{\ConstantTok}[1]{\textcolor[rgb]{0.56,0.35,0.01}{#1}}
\newcommand{\ControlFlowTok}[1]{\textcolor[rgb]{0.13,0.29,0.53}{\textbf{#1}}}
\newcommand{\DataTypeTok}[1]{\textcolor[rgb]{0.13,0.29,0.53}{#1}}
\newcommand{\DecValTok}[1]{\textcolor[rgb]{0.00,0.00,0.81}{#1}}
\newcommand{\DocumentationTok}[1]{\textcolor[rgb]{0.56,0.35,0.01}{\textbf{\textit{#1}}}}
\newcommand{\ErrorTok}[1]{\textcolor[rgb]{0.64,0.00,0.00}{\textbf{#1}}}
\newcommand{\ExtensionTok}[1]{#1}
\newcommand{\FloatTok}[1]{\textcolor[rgb]{0.00,0.00,0.81}{#1}}
\newcommand{\FunctionTok}[1]{\textcolor[rgb]{0.13,0.29,0.53}{\textbf{#1}}}
\newcommand{\ImportTok}[1]{#1}
\newcommand{\InformationTok}[1]{\textcolor[rgb]{0.56,0.35,0.01}{\textbf{\textit{#1}}}}
\newcommand{\KeywordTok}[1]{\textcolor[rgb]{0.13,0.29,0.53}{\textbf{#1}}}
\newcommand{\NormalTok}[1]{#1}
\newcommand{\OperatorTok}[1]{\textcolor[rgb]{0.81,0.36,0.00}{\textbf{#1}}}
\newcommand{\OtherTok}[1]{\textcolor[rgb]{0.56,0.35,0.01}{#1}}
\newcommand{\PreprocessorTok}[1]{\textcolor[rgb]{0.56,0.35,0.01}{\textit{#1}}}
\newcommand{\RegionMarkerTok}[1]{#1}
\newcommand{\SpecialCharTok}[1]{\textcolor[rgb]{0.81,0.36,0.00}{\textbf{#1}}}
\newcommand{\SpecialStringTok}[1]{\textcolor[rgb]{0.31,0.60,0.02}{#1}}
\newcommand{\StringTok}[1]{\textcolor[rgb]{0.31,0.60,0.02}{#1}}
\newcommand{\VariableTok}[1]{\textcolor[rgb]{0.00,0.00,0.00}{#1}}
\newcommand{\VerbatimStringTok}[1]{\textcolor[rgb]{0.31,0.60,0.02}{#1}}
\newcommand{\WarningTok}[1]{\textcolor[rgb]{0.56,0.35,0.01}{\textbf{\textit{#1}}}}
\usepackage{longtable,booktabs,array}
\usepackage{calc} % for calculating minipage widths
% Correct order of tables after \paragraph or \subparagraph
\usepackage{etoolbox}
\makeatletter
\patchcmd\longtable{\par}{\if@noskipsec\mbox{}\fi\par}{}{}
\makeatother
% Allow footnotes in longtable head/foot
\IfFileExists{footnotehyper.sty}{\usepackage{footnotehyper}}{\usepackage{footnote}}
\makesavenoteenv{longtable}
\usepackage{graphicx}
\makeatletter
\def\maxwidth{\ifdim\Gin@nat@width>\linewidth\linewidth\else\Gin@nat@width\fi}
\def\maxheight{\ifdim\Gin@nat@height>\textheight\textheight\else\Gin@nat@height\fi}
\makeatother
% Scale images if necessary, so that they will not overflow the page
% margins by default, and it is still possible to overwrite the defaults
% using explicit options in \includegraphics[width, height, ...]{}
\setkeys{Gin}{width=\maxwidth,height=\maxheight,keepaspectratio}
% Set default figure placement to htbp
\makeatletter
\def\fps@figure{htbp}
\makeatother
\setlength{\emergencystretch}{3em} % prevent overfull lines
\providecommand{\tightlist}{%
  \setlength{\itemsep}{0pt}\setlength{\parskip}{0pt}}
\setcounter{secnumdepth}{5}
\usepackage{ctex}

%\usepackage{xltxtra} % XeLaTeX的一些额外符号
% 设置中文字体
%\setCJKmainfont[BoldFont={黑体},ItalicFont={楷体}]{新宋体}

% 设置边距
\usepackage{geometry}
\geometry{%
  left=2.0cm, right=2.0cm, top=3.5cm, bottom=2.5cm} 

\usepackage{amsthm,mathrsfs}
\usepackage{booktabs}
\usepackage{longtable}
\makeatletter
\def\thm@space@setup{%
  \thm@preskip=8pt plus 2pt minus 4pt
  \thm@postskip=\thm@preskip
}
\makeatother
\ifLuaTeX
  \usepackage{selnolig}  % disable illegal ligatures
\fi
\usepackage[]{biblatex}
\usepackage{bookmark}
\IfFileExists{xurl.sty}{\usepackage{xurl}}{} % add URL line breaks if available
\urlstyle{same}
\hypersetup{
  hidelinks,
  pdfcreator={LaTeX via pandoc}}

\author{}
\date{\vspace{-2.5em}}

\begin{document}

{
\setcounter{tocdepth}{2}
\tableofcontents
}
\section{R语言教程-官方管道符和magrittr管道符区别概览}\label{rux8bedux8a00ux6559ux7a0b-ux5b98ux65b9ux7ba1ux9053ux7b26ux548cmagrittrux7ba1ux9053ux7b26ux533aux522bux6982ux89c8}

\begin{quote}
第73期

本文是作为学习笔记,学习R语言的官方管道符\texttt{\textbar{}\textgreater{}(4.1版本之后引入)}和magrittr包的管道符\texttt{\%\textgreater{}\%}(目前我比较常用)的区别

总结一下:从功能性,简洁性来看,\textbf{完全建议使用magrittr包的\%\textgreater\%管道符。}

讨论来源:\url{https://stackoverflow.com/questions/67633022/what-are-the-differences-between-rs-native-pipe-and-the-magrittr-pipe}
\end{quote}

\subsection{概述}\label{ux6982ux8ff0}

\begin{longtable}[]{@{}
  >{\raggedright\arraybackslash}p{(\columnwidth - 4\tabcolsep) * \real{0.2477}}
  >{\raggedright\arraybackslash}p{(\columnwidth - 4\tabcolsep) * \real{0.3761}}
  >{\raggedright\arraybackslash}p{(\columnwidth - 4\tabcolsep) * \real{0.3670}}@{}}
\toprule\noalign{}
\begin{minipage}[b]{\linewidth}\raggedright
\textbf{Topic}
\end{minipage} & \begin{minipage}[b]{\linewidth}\raggedright
\textbf{Magrittr \emph{2.0.3}}
\end{minipage} & \begin{minipage}[b]{\linewidth}\raggedright
\textbf{官方\emph{4.3.0}}
\end{minipage} \\
\midrule\noalign{}
\endhead
\bottomrule\noalign{}
\endlastfoot
\textbf{管道符} & \texttt{\%\textgreater{}\%} \texttt{\%\textless{}\textgreater{}\%} \texttt{\%\$\%} \texttt{\%!\textgreater{}\%} \texttt{\%T\textgreater{}\%} & \texttt{\textbar{}\textgreater{}} (since 4.1.0) \\
\textbf{函数调用差异} & \texttt{1:3\ \%\textgreater{}\%\ sum()} & \texttt{1:3\ \textbar{}\textgreater{}\ sum()} \\
& \texttt{1:3\ \%\textgreater{}\%\ sum} & 官方的需要加括号 \\
& \texttt{1:3\ \%\textgreater{}\%\ \textasciigrave{}+\textasciigrave{}(4)} & 部分函数不支持 \\
是否可以作为第一参数插入 & \texttt{mtcars\ \%\textgreater{}\%\ lm(formula\ =\ mpg\ \textasciitilde{}\ disp)} & \texttt{mtcars\ \textbar{}\textgreater{}\ lm(formula\ =\ mpg\ \textasciitilde{}\ disp)} \\
占位符号差异 & \texttt{.} & \texttt{\_} (since 4.2.0) \\
& \texttt{mtcars\ \%\textgreater{}\%\ lm(mpg\ \textasciitilde{}\ disp,\ data\ =\ .\ )} & \texttt{mtcars\ \textbar{}\textgreater{}\ lm(mpg\ \textasciitilde{}\ disp,\ data\ =\ \_\ )} \\
& \texttt{mtcars\ \%\textgreater{}\%\ lm(mpg\ \textasciitilde{}\ disp,\ .\ )} & 需要提供参数名 data \\
& \texttt{1:3\ \%\textgreater{}\%\ setNames(.,\ .)} & 只能替代单次 \\
& \texttt{1:3\ \%\textgreater{}\%\ \{sum(sqrt(.))\}} & 不允许嵌套调用 \\
提取调用(都比较符合) & \begin{minipage}[t]{\linewidth}\raggedright
\texttt{mtcars\ \%\textgreater{}\%\ .\$cyl}\strut \\
\texttt{mtcars\ \%\textgreater{}\%\ \{.\$cyl{[}{[}3{]}{]}\}} or\\
\texttt{mtcars\ \%\$\%\ cyl{[}{[}3{]}{]}}\strut
\end{minipage} & \begin{minipage}[t]{\linewidth}\raggedright
\texttt{mtcars\ \textbar{}\textgreater{}\ \_\$cyl} (since 4.3.0)\\
\texttt{mtcars\ \textbar{}\textgreater{}\ \_\$cyl{[}{[}3{]}{]}}\strut \\
\strut
\end{minipage} \\
是否可以创建函数 & \texttt{top6\ \textless{}-\ .\ \%\textgreater{}\%\ sort()\ \%\textgreater{}\%\ tail()} & 无法实现 \\
\textbf{运行速度} & 由于函数调用多而运行慢 & 运行速度较快 \\
\end{longtable}

\subsection{详细说明}\label{ux8be6ux7ec6ux8bf4ux660e}

\subsubsection{官方管道符使用需要有括号}\label{ux5b98ux65b9ux7ba1ux9053ux7b26ux4f7fux7528ux9700ux8981ux6709ux62ecux53f7}

\begin{Shaded}
\begin{Highlighting}[]
\FunctionTok{library}\NormalTok{(magrittr)}
\end{Highlighting}
\end{Shaded}

\begin{Shaded}
\begin{Highlighting}[]
\DecValTok{1}\SpecialCharTok{:}\DecValTok{3} \SpecialCharTok{|\textgreater{}}\NormalTok{ sum}
\end{Highlighting}
\end{Shaded}

\begin{verbatim}
## Error: The pipe operator requires a function call as RHS (<text>:1:8)
\end{verbatim}

\begin{Shaded}
\begin{Highlighting}[]
\DecValTok{1}\SpecialCharTok{:}\DecValTok{3} \SpecialCharTok{|\textgreater{}} \FunctionTok{sum}\NormalTok{()}
\end{Highlighting}
\end{Shaded}

\begin{verbatim}
## [1] 6
\end{verbatim}

\begin{Shaded}
\begin{Highlighting}[]
\DecValTok{1}\SpecialCharTok{:}\DecValTok{3} \SpecialCharTok{|\textgreater{}} \FunctionTok{approxfun}\NormalTok{(}\DecValTok{1}\SpecialCharTok{:}\DecValTok{3}\NormalTok{, }\DecValTok{4}\SpecialCharTok{:}\DecValTok{6}\NormalTok{)()}
\end{Highlighting}
\end{Shaded}

\begin{verbatim}
## [1] 4 5 6
\end{verbatim}

\begin{Shaded}
\begin{Highlighting}[]
\DecValTok{1}\SpecialCharTok{:}\DecValTok{3} \SpecialCharTok{\%\textgreater{}\%}\NormalTok{ sum}
\end{Highlighting}
\end{Shaded}

\begin{verbatim}
## [1] 6
\end{verbatim}

\begin{Shaded}
\begin{Highlighting}[]
\DecValTok{1}\SpecialCharTok{:}\DecValTok{3} \SpecialCharTok{\%\textgreater{}\%} \FunctionTok{sum}\NormalTok{()}
\end{Highlighting}
\end{Shaded}

\begin{verbatim}
## [1] 6
\end{verbatim}

\begin{Shaded}
\begin{Highlighting}[]
\DecValTok{1}\SpecialCharTok{:}\DecValTok{3} \SpecialCharTok{\%\textgreater{}\%} \FunctionTok{approxfun}\NormalTok{(}\DecValTok{1}\SpecialCharTok{:}\DecValTok{3}\NormalTok{, }\DecValTok{4}\SpecialCharTok{:}\DecValTok{6}\NormalTok{) }
\end{Highlighting}
\end{Shaded}

\begin{verbatim}
## Error in if (is.na(method)) stop("invalid interpolation method"): the condition has length > 1
\end{verbatim}

\begin{Shaded}
\begin{Highlighting}[]
\DecValTok{1}\SpecialCharTok{:}\DecValTok{3} \SpecialCharTok{\%\textgreater{}\%} \FunctionTok{approxfun}\NormalTok{(}\DecValTok{1}\SpecialCharTok{:}\DecValTok{3}\NormalTok{, }\DecValTok{4}\SpecialCharTok{:}\DecValTok{6}\NormalTok{)()}
\end{Highlighting}
\end{Shaded}

\begin{verbatim}
## [1] 4 5 6
\end{verbatim}

\subsubsection{官方管道符一些函数无法调用,可使用::调用或加括号调用}\label{ux5b98ux65b9ux7ba1ux9053ux7b26ux4e00ux4e9bux51fdux6570ux65e0ux6cd5ux8c03ux7528ux53efux4f7fux7528ux8c03ux7528ux6216ux52a0ux62ecux53f7ux8c03ux7528}

\begin{Shaded}
\begin{Highlighting}[]
\DecValTok{1}\SpecialCharTok{:}\DecValTok{3} \SpecialCharTok{|\textgreater{}} \StringTok{\textasciigrave{}}\AttributeTok{+}\StringTok{\textasciigrave{}}\NormalTok{(}\DecValTok{4}\NormalTok{)}
\end{Highlighting}
\end{Shaded}

\begin{verbatim}
## Error: function '+' not supported in RHS call of a pipe (<text>:1:8)
\end{verbatim}

\begin{Shaded}
\begin{Highlighting}[]
\DecValTok{1}\SpecialCharTok{:}\DecValTok{3} \SpecialCharTok{|\textgreater{}}\NormalTok{ (}\StringTok{\textasciigrave{}}\AttributeTok{+}\StringTok{\textasciigrave{}}\NormalTok{)(}\DecValTok{4}\NormalTok{)}
\end{Highlighting}
\end{Shaded}

\begin{verbatim}
## [1] 5 6 7
\end{verbatim}

\begin{Shaded}
\begin{Highlighting}[]
\DecValTok{1}\SpecialCharTok{:}\DecValTok{3} \SpecialCharTok{|\textgreater{}}\NormalTok{ base}\SpecialCharTok{::}\StringTok{\textasciigrave{}}\AttributeTok{+}\StringTok{\textasciigrave{}}\NormalTok{(}\DecValTok{4}\NormalTok{)}
\end{Highlighting}
\end{Shaded}

\begin{verbatim}
## [1] 5 6 7
\end{verbatim}

\subsubsection{官方管道符的占位符一定需要加上参数名}\label{ux5b98ux65b9ux7ba1ux9053ux7b26ux7684ux5360ux4f4dux7b26ux4e00ux5b9aux9700ux8981ux52a0ux4e0aux53c2ux6570ux540d}

\begin{Shaded}
\begin{Highlighting}[]
\DecValTok{2} \SpecialCharTok{|\textgreater{}} \FunctionTok{setdiff}\NormalTok{(}\DecValTok{1}\SpecialCharTok{:}\DecValTok{3}\NormalTok{, \_)}
\end{Highlighting}
\end{Shaded}

\begin{verbatim}
## Error: pipe placeholder can only be used as a named argument (<text>:1:6)
\end{verbatim}

\begin{Shaded}
\begin{Highlighting}[]
\DecValTok{2} \SpecialCharTok{|\textgreater{}} \FunctionTok{setdiff}\NormalTok{(}\DecValTok{1}\SpecialCharTok{:}\DecValTok{3}\NormalTok{, }\AttributeTok{y =}\NormalTok{ \_)}
\end{Highlighting}
\end{Shaded}

\begin{verbatim}
## [1] 1 3
\end{verbatim}

\subsubsection{可变参数函数使用官方管道符}\label{ux53efux53d8ux53c2ux6570ux51fdux6570ux4f7fux7528ux5b98ux65b9ux7ba1ux9053ux7b26}

可变参数函数是那些能够接受可变数量的参数的函数,需要添加`.`做参数名

\begin{Shaded}
\begin{Highlighting}[]
\StringTok{"b"} \SpecialCharTok{|\textgreater{}}  \FunctionTok{paste}\NormalTok{(}\StringTok{"a"}\NormalTok{, \_, }\StringTok{"c"}\NormalTok{)}
\end{Highlighting}
\end{Shaded}

\begin{verbatim}
## Error: pipe placeholder can only be used as a named argument (<text>:1:9)
\end{verbatim}

\begin{Shaded}
\begin{Highlighting}[]
\StringTok{"b"} \SpecialCharTok{|\textgreater{}}  \FunctionTok{paste}\NormalTok{(}\StringTok{"a"}\NormalTok{, }\AttributeTok{. =}\NormalTok{ \_, }\StringTok{"c"}\NormalTok{)}
\end{Highlighting}
\end{Shaded}

\begin{verbatim}
## [1] "a b c"
\end{verbatim}

\subsubsection{官方管道符的参数占位符只能调用一次}\label{ux5b98ux65b9ux7ba1ux9053ux7b26ux7684ux53c2ux6570ux5360ux4f4dux7b26ux53eaux80fdux8c03ux7528ux4e00ux6b21}

\begin{Shaded}
\begin{Highlighting}[]
\DecValTok{1}\SpecialCharTok{:}\DecValTok{3} \SpecialCharTok{|\textgreater{}} \FunctionTok{setNames}\NormalTok{(}\AttributeTok{nm =}\NormalTok{ \_)}
\end{Highlighting}
\end{Shaded}

\begin{verbatim}
## 1 2 3 
## 1 2 3
\end{verbatim}

\begin{Shaded}
\begin{Highlighting}[]
\DecValTok{1}\SpecialCharTok{:}\DecValTok{3} \SpecialCharTok{|\textgreater{}} \FunctionTok{setNames}\NormalTok{(}\AttributeTok{object =}\NormalTok{ \_, }\AttributeTok{nm =}\NormalTok{ \_)}
\end{Highlighting}
\end{Shaded}

\begin{verbatim}
## Error: pipe placeholder may only appear once (<text>:1:8)
\end{verbatim}

\subsubsection{官方管道符不支持嵌套调用}\label{ux5b98ux65b9ux7ba1ux9053ux7b26ux4e0dux652fux6301ux5d4cux5957ux8c03ux7528}

\begin{Shaded}
\begin{Highlighting}[]
\DecValTok{1}\SpecialCharTok{:}\DecValTok{3} \SpecialCharTok{|\textgreater{}} \FunctionTok{sum}\NormalTok{(}\FunctionTok{sqrt}\NormalTok{(}\AttributeTok{x=}\NormalTok{\_))}
\end{Highlighting}
\end{Shaded}

\begin{verbatim}
## Error: invalid use of pipe placeholder (<text>:1:0)
\end{verbatim}

\begin{Shaded}
\begin{Highlighting}[]
\DecValTok{1}\SpecialCharTok{:}\DecValTok{3} \SpecialCharTok{|\textgreater{}}\NormalTok{ (\textbackslash{}(.) }\FunctionTok{sum}\NormalTok{(}\FunctionTok{sqrt}\NormalTok{(.)))()}
\end{Highlighting}
\end{Shaded}

\begin{verbatim}
## [1] 4.146264
\end{verbatim}

\begin{Shaded}
\begin{Highlighting}[]
\CommentTok{\#[1] 4.146264}
\end{Highlighting}
\end{Shaded}

\subsubsection{提取调用}\label{ux63d0ux53d6ux8c03ux7528}

自 4.3.0 起的试验性功能。现在,占位符 \_ 也可以作为提取调用的第一个参数,用于正向管道 \textbar\textgreater{} 表达式的 rhs 中,例如 \_\$coef。更广泛地说,它可以用作提取链的首参数,例如 \_\$coef{[}{[}2{]}{]}

但\%\textgreater\%一直支持

\subsubsection{创建函数}\label{ux521bux5efaux51fdux6570}

\%\textgreater\%支持直接创建新函数,官方管道符不支持

\begin{Shaded}
\begin{Highlighting}[]
\NormalTok{top6 }\OtherTok{\textless{}{-}}\NormalTok{ . }\SpecialCharTok{\%\textgreater{}\%} \FunctionTok{sort}\NormalTok{() }\SpecialCharTok{\%\textgreater{}\%} \FunctionTok{tail}\NormalTok{()}
\FunctionTok{top6}\NormalTok{(}\FunctionTok{c}\NormalTok{(}\DecValTok{1}\SpecialCharTok{:}\DecValTok{10}\NormalTok{,}\DecValTok{10}\SpecialCharTok{:}\DecValTok{1}\NormalTok{))}
\end{Highlighting}
\end{Shaded}

\begin{verbatim}
## [1]  8  8  9  9 10 10
\end{verbatim}

\begin{Shaded}
\begin{Highlighting}[]
\CommentTok{\#[1]  8  8  9  9 10 10}
\end{Highlighting}
\end{Shaded}


\printbibliography

\end{document}
